\documentclass[12]{report}
\usepackage{fullpage}

\begin{document}

\centerline{\bf\large Glossary of terms for Visual Inference}

\begin{itemize}
\item $H_0$ null hypothesis, eg X, Y are independent \\
\item $H_a$  \\
\item test statistic  plot of observed data  \\
\item sampling distribution  \\
\item Significance level $\alpha$  $\alpha \in (0,1)$ is the threshold at which we will reject the null hypothesis, should the $p$-value fall under it, usually $\alpha=0.05$.\\
\item test procedure  a set of $K$ independent judges evaluates a lineup of size $m$ to come to a conclusion whether to reject $H_0$.\\
\item $p$-value  probability that $x$ or more of $K$ observers pick the data plot  under $H_0$. \\
\item Power  probability to reject the null hypothesis\\
\item Type I error  probability to reject the null hypothesis if it is true\\
\item Type II error  probability to fail to reject the null hypothesis if it is false\\

\item \textbf{Sampling Distribution}: The sampling distribution is the distribution of the sample statistic based on all possible samples of sample size $n$ from the population. So we have a population of size 30 and we take samples of size 5, there are $c^{30}_5$ possible samples from the population. If we calculate the mean of all the samples and draw a histogram, the histogram gives the sampling distribution of the sample mean.
\item \textbf{Null Distribution}: The null distribution corresponds to the probability distribution of the test statistic when the null hypothesis is true. So if we want to test for the means, \\

$H_0: \mu = 0$ vs $H_a: \mu \ne 0$\\

the test statistic is given by \\

$$t=\frac{\bar{X} - 0}{S/\sqrt{n}}$$\\

The distribution of the test statistic $t$ when $H_0$ is true is a central $t$ distribution with (n - 1) degrees of freedom. This is the null distribution of the test statistic $t$.
\end{itemize}
\end{document}