%% =================chapter 4 starts here ===========================
%
\chapter{Teaching Visual Inference}\label{ch:teaching} This section focuses on the development of the teaching materials for undergraduate courses. From various experiments, it has been observed that significant number of evaluations are correct even though the education level is not even graduate. Even if the viewers are well educated they have no idea what the experiment is about. But still their evaluations indicates strong power of statistical inference. This suggests that Statistical Inference could be introduced to the students earlier than what we usually do now since it does not require a lot of technical knowledge. Visual Inference technique could be their first step in learning statistical inference. This would make a strong mindset about the core of statistical inference with less effort but with lots of fun.   


\section{Developing Teaching Materials} This section will describe the procedure in developing the teaching materials. Assuming that the student would be seeing statistical inference for the first time, the visual inference material has to be developed such that it explains each term used in statistical inference vividly.

\section{Developing materials which describes the analogies} In this section we plan to develop the material which will compare the visual inference to classical inference. So we need to review the classical inference by using examples which will describe both the procedures.

\section{Developing materials to test the students} In this section we plan to develop materials which will test the students on the concepts of visual and classical inference.

%\section{Influence of Demographical Factors on Validation of Lineup} We collected several demographical information through our turk experiments. We already observed that gender does not have significant effect on successful evaluation of the lineup plot. But the age or the education level seem to have some effect. This section will analyze the extent of this influence.
%
%\section{Quality of Turk Data} Can we depend on Turk data? This question will be addressed in this section.
%
%\section{Comparison of Turk Data with the Data obtained from More Trained Participants} Turk users are not necessarily trained on visual or statistical plots. It is interesting to know how their performances compare with the performances of the people who are generally trained on statistical plots and data visualization. We have feedbacks on lineup plots from later group of people. We intend to present a comparison study between these two types of evaluators.

